\section{Introduction\label{chapter1}}
%Briefly introduce \textsc{nmf}, applications
Non-negative matrix factorization (\textsc{nmf}) is a matrix decomposition technique that approximates a multivariate data matrix by two lower dimensional non-negative matrices as follows:
\begin{equation}
  vv
\end{equation}
As \textsc{nmf} only allows additive, non-subtractive combination of matrix factors, it is applicable to an extensive range of domain \chenc{sorry but I do not understand}. \citet{lee1999learning} suggest that \textsc{nmf} is useful for image processing problems including facial recognition. Specifically, \textsc{nmf} generates two matrices $W$ and $H$ which are often referred as the basis images and weights. This is because the observed image $V$ is approximated by a linear combination of $W$ and $H$. This property also distinguishes \textsc{nmf} from other traditional image processing methods such as principal components analysis (\textsc{pca}) and $K$-means clustering \chenc{$K$ means is not only a image processing technique, delete}. \citet{guillamet2002non} demonstrate that \textsc{nmf} is more robust to corrupted images than \textsc{pca}.
Moreover, \textsc{nmf} is also applicable to text mining such as semantic analysis. More generally, \textsc{nmf} is useful to discover semantic features of an article by counting the frequency of each word than approximating the document from a subset of a large array of features \citep{lee1999learning}.

Researchers proposed many \textsc{nmf} algorithms. \citet{lee2001algorithms} first proposes ``multiplicative update rules'' to minimise Euclidean distance or Kullback-Leibler divergence between the original matrix and its approximation. Although this algorithm is easy to implement and have reasonable convergent rate \citep{lee2001algorithms}, it may fail on seriously corrupted dataset which violates its assumption of Gaussian noise or Poisson noise, respectively \citep{guan2017truncated}.  To improve the robustness of \textsc{nmf}, many methods have been proposed. Lam ????? \chenc{no bib file}(2008) proposes ${L_1}$-norm based \textsc{nmf} to model noisy data by a Laplace distribution which is less sensitive to outliers. However, as $L_1$-norm is not differentiable at zero, the optimization procedure is computationally expensive. \citet{kong2011robust} proposed an \textsc{nmf} algorithm using $L_{21}$-norm loss function which is robust to outliers. The updating rules used in $L_{21}$-norm \textsc{nmf}, however, converge slowly because of a continual use of the power method \citep{guan2017truncated}.

In practice, face images could be easily corrupted during data collection by large magnitude noise. Corruption may result from lighting environment, facial expression or facial details. An \textsc{nmf} algorithm that is robust to large noise is desired for real-world application. Therefore, the objective of this project is to analyse the robustness of \textsc{nmf} algorithms on corrupted dataset. We implement two \textsc{nmf} algorithms on real face image datasets, \textsc{orl} dataset and Extended YaleB dataset. The face images are contaminated by artificial noises.

Plan:
...



